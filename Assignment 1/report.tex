\PassOptionsToPackage{unicode=true}{hyperref} % options for packages loaded elsewhere
\PassOptionsToPackage{hyphens}{url}
%
\documentclass[]{article}
\usepackage{lmodern}
\usepackage{amssymb,amsmath}
\usepackage{ifxetex,ifluatex}
\usepackage{fixltx2e} % provides \textsubscript
\ifnum 0\ifxetex 1\fi\ifluatex 1\fi=0 % if pdftex
  \usepackage[T1]{fontenc}
  \usepackage[utf8]{inputenc}
  \usepackage{textcomp} % provides euro and other symbols
\else % if luatex or xelatex
  \usepackage{unicode-math}
  \defaultfontfeatures{Ligatures=TeX,Scale=MatchLowercase}
\fi
% use upquote if available, for straight quotes in verbatim environments
\IfFileExists{upquote.sty}{\usepackage{upquote}}{}
% use microtype if available
\IfFileExists{microtype.sty}{%
\usepackage[]{microtype}
\UseMicrotypeSet[protrusion]{basicmath} % disable protrusion for tt fonts
}{}
\IfFileExists{parskip.sty}{%
\usepackage{parskip}
}{% else
\setlength{\parindent}{0pt}
\setlength{\parskip}{6pt plus 2pt minus 1pt}
}
\usepackage{hyperref}
\hypersetup{
            pdftitle={Practice 0: Naïve Bayes},
            pdfauthor={Javier Esteban Aragoneses, Mauricio Marcos, Danyu Zhang, Daniel Alonso},
            pdfborder={0 0 0},
            breaklinks=true}
\urlstyle{same}  % don't use monospace font for urls
\usepackage[margin=1in]{geometry}
\usepackage{color}
\usepackage{fancyvrb}
\newcommand{\VerbBar}{|}
\newcommand{\VERB}{\Verb[commandchars=\\\{\}]}
\DefineVerbatimEnvironment{Highlighting}{Verbatim}{commandchars=\\\{\}}
% Add ',fontsize=\small' for more characters per line
\usepackage{framed}
\definecolor{shadecolor}{RGB}{248,248,248}
\newenvironment{Shaded}{\begin{snugshade}}{\end{snugshade}}
\newcommand{\AlertTok}[1]{\textcolor[rgb]{0.94,0.16,0.16}{#1}}
\newcommand{\AnnotationTok}[1]{\textcolor[rgb]{0.56,0.35,0.01}{\textbf{\textit{#1}}}}
\newcommand{\AttributeTok}[1]{\textcolor[rgb]{0.77,0.63,0.00}{#1}}
\newcommand{\BaseNTok}[1]{\textcolor[rgb]{0.00,0.00,0.81}{#1}}
\newcommand{\BuiltInTok}[1]{#1}
\newcommand{\CharTok}[1]{\textcolor[rgb]{0.31,0.60,0.02}{#1}}
\newcommand{\CommentTok}[1]{\textcolor[rgb]{0.56,0.35,0.01}{\textit{#1}}}
\newcommand{\CommentVarTok}[1]{\textcolor[rgb]{0.56,0.35,0.01}{\textbf{\textit{#1}}}}
\newcommand{\ConstantTok}[1]{\textcolor[rgb]{0.00,0.00,0.00}{#1}}
\newcommand{\ControlFlowTok}[1]{\textcolor[rgb]{0.13,0.29,0.53}{\textbf{#1}}}
\newcommand{\DataTypeTok}[1]{\textcolor[rgb]{0.13,0.29,0.53}{#1}}
\newcommand{\DecValTok}[1]{\textcolor[rgb]{0.00,0.00,0.81}{#1}}
\newcommand{\DocumentationTok}[1]{\textcolor[rgb]{0.56,0.35,0.01}{\textbf{\textit{#1}}}}
\newcommand{\ErrorTok}[1]{\textcolor[rgb]{0.64,0.00,0.00}{\textbf{#1}}}
\newcommand{\ExtensionTok}[1]{#1}
\newcommand{\FloatTok}[1]{\textcolor[rgb]{0.00,0.00,0.81}{#1}}
\newcommand{\FunctionTok}[1]{\textcolor[rgb]{0.00,0.00,0.00}{#1}}
\newcommand{\ImportTok}[1]{#1}
\newcommand{\InformationTok}[1]{\textcolor[rgb]{0.56,0.35,0.01}{\textbf{\textit{#1}}}}
\newcommand{\KeywordTok}[1]{\textcolor[rgb]{0.13,0.29,0.53}{\textbf{#1}}}
\newcommand{\NormalTok}[1]{#1}
\newcommand{\OperatorTok}[1]{\textcolor[rgb]{0.81,0.36,0.00}{\textbf{#1}}}
\newcommand{\OtherTok}[1]{\textcolor[rgb]{0.56,0.35,0.01}{#1}}
\newcommand{\PreprocessorTok}[1]{\textcolor[rgb]{0.56,0.35,0.01}{\textit{#1}}}
\newcommand{\RegionMarkerTok}[1]{#1}
\newcommand{\SpecialCharTok}[1]{\textcolor[rgb]{0.00,0.00,0.00}{#1}}
\newcommand{\SpecialStringTok}[1]{\textcolor[rgb]{0.31,0.60,0.02}{#1}}
\newcommand{\StringTok}[1]{\textcolor[rgb]{0.31,0.60,0.02}{#1}}
\newcommand{\VariableTok}[1]{\textcolor[rgb]{0.00,0.00,0.00}{#1}}
\newcommand{\VerbatimStringTok}[1]{\textcolor[rgb]{0.31,0.60,0.02}{#1}}
\newcommand{\WarningTok}[1]{\textcolor[rgb]{0.56,0.35,0.01}{\textbf{\textit{#1}}}}
\usepackage{graphicx,grffile}
\makeatletter
\def\maxwidth{\ifdim\Gin@nat@width>\linewidth\linewidth\else\Gin@nat@width\fi}
\def\maxheight{\ifdim\Gin@nat@height>\textheight\textheight\else\Gin@nat@height\fi}
\makeatother
% Scale images if necessary, so that they will not overflow the page
% margins by default, and it is still possible to overwrite the defaults
% using explicit options in \includegraphics[width, height, ...]{}
\setkeys{Gin}{width=\maxwidth,height=\maxheight,keepaspectratio}
\setlength{\emergencystretch}{3em}  % prevent overfull lines
\providecommand{\tightlist}{%
  \setlength{\itemsep}{0pt}\setlength{\parskip}{0pt}}
\setcounter{secnumdepth}{0}
% Redefines (sub)paragraphs to behave more like sections
\ifx\paragraph\undefined\else
\let\oldparagraph\paragraph
\renewcommand{\paragraph}[1]{\oldparagraph{#1}\mbox{}}
\fi
\ifx\subparagraph\undefined\else
\let\oldsubparagraph\subparagraph
\renewcommand{\subparagraph}[1]{\oldsubparagraph{#1}\mbox{}}
\fi

% set default figure placement to htbp
\makeatletter
\def\fps@figure{htbp}
\makeatother


\title{Practice 0: Naïve Bayes}
\author{Javier Esteban Aragoneses, Mauricio Marcos, Danyu Zhang, Daniel Alonso}
\date{February 7th, 2021}

\begin{document}
\maketitle

\hypertarget{dataset-selection}{%
\section{Dataset selection}\label{dataset-selection}}

In this Project, we will use a database obtained from the World Bank
Databank, specifically the World Development Indicators database. It
consists of economics, education, energy use and population specific
information. So, in this case, our goal is to predict, using demographic
variables, a development measure called ``Human Development Index''
(from now on, ``HDI'').

We have selected a country list dataset where we predict HDI category
per country for this assignment. We replace \emph{very high} and
\emph{high} HDI with only the tag \textbf{high}. We apply the same
treatment to \emph{medium} and \emph{low} HDI with only the tag
\textbf{low}. This simplifies the analysis greatly, and improves
accuracy.

Our target variable here will be \emph{hdi\_cat}.

Variables:

We have \#\# number of features in our dataset, each one expresses the
following:

\begin{itemize}
\tightlist
\item
  \textbf{foreign\_inv\_inflows}: Foreign direct investment, net inflows
  (BoP, current US)
\item
  \textbf{exports\_perc\_gdp}: Exports of goods and services (as a \% of
  GDP)
\item
  \textbf{inflation\_perc}: Inflation, consumer prices (annual \%)
\item
  \textbf{education\_years}: Compulsory education, duration (years)
\item
  \textbf{education\_perc\_gdp}: Government expenditure on education,
  total (as a \% of GDP)
\item
  \textbf{gds\_perc\_gdp}: Gross domestic savings (as a \% of GDP)
\item
  \textbf{gross\_savings\_perc\_gdp}: Gross savings (as a \% of GDP)
\item
  \textbf{int\_tourism\_arrivals}: International tourism, number of
  arrivals
\item
  \textbf{int\_tourism\_receipts}: International tourism, receipts (in
  current US)
\item
  \textbf{perc\_internet\_users}: Individuals using the Internet (as a
  \% of population)
\item
  \textbf{access\_to\_electricity}: Access to electricity (\% of
  population)
\item
  \textbf{agricultural\_land}: Agricultural land (\% of land area)
\item
  \textbf{birth\_rate}: Birth rate, crude (per 1,000 people)
\item
  \textbf{gne}: Gross national expenditure (\% of GDP)
\item
  \textbf{mobile\_subscriptions}: Mobile cellular subscriptions (per 100
  people)
\item
  \textbf{infant\_mort\_rate}: Mortality rate, infant (per 1,000 live
  births)
\item
  \textbf{sex\_ratio}: Sex ratio at birth (male births per female
  births)
\item
  \textbf{greenhouse\_gas\_em}: Total greenhouse gas emissions (kt of
  CO2 equivalent)
\item
  \textbf{urban\_pop\_perc}: Urban population (\% of total population)
\end{itemize}

\hypertarget{train-test-split}{%
\section{Train-Test split}\label{train-test-split}}

We create a train-test split with 70\% train and 30\% test.

\newpage

\hypertarget{creating-a-nauxefve-bayes-model}{%
\section{Creating a Naïve Bayes
model}\label{creating-a-nauxefve-bayes-model}}

In this case, we will use the Naïve Bayes Classificator, to predict
weather a country will be in the group of \emph{very high}, \emph{high},
\emph{medium} or \emph{low} HDI.

The Naïve Bayes is sometimes a good model to predict a classification
through the study of features, as it takes into account prior
probabilities, to estimate the prediction. So first, the algorithm will
find the prior probabilities (given by the percentaje of each category
in the training sample set) and then will calculate the likelihoods,
using the Bayes Rule. But when doing so, it is making two big
assumptions that may arise in problems later:

\begin{itemize}
\tightlist
\item
  The effect of the value of a predictor (or features) on a given class
  (high or low HDI) is independent of the values of other predictors.
\item
  Each feature makes an equal contribution to the outcome.
\end{itemize}

This is why, after calculating every conditional probability given each
feature value, it multiplies all the probabilities to find the final
probability of each observation for each category.

In other words, according to our first assumption the percentage of
urban population in a country is independent from the life expectancy of
the country in a given class, and the total greenhouse gas emissions is
also independent from both of them. Then, our second assumptions says
that none of the features is meaningless to predict our categories, and
that all of them gives equal contribution to the predictions. We will
see that this is not the case.

Finally, the algorithm will calculate the probability of being of a
specific category and then will create a classifier. For this, we find
the probability of given set of inputs for all possible values of the
class variable (very high, high, medium, low, very low) and pick up the
output with the maximum probability.

As we can see, the model takes as prior probabilities for each
cateogory, the percentaje of observations from the training set of each
category. In this case, we have that prior probability of being ``low
HDI'': 0.398 and the prior probability of a ``high HDI'' is 0.602. Then,
it calculates the frequencies for each attribute against the target, and
then the likelihoods in a table. In this case the likelihood is the
probability of the predictor given a specific class. And then, it
calculates the posterior probabilities.

\hypertarget{predicting-hdi-category}{%
\section{Predicting HDI category}\label{predicting-hdi-category}}

Performing the prediction:

Calculating a confidence interval for each prediction:

Confusion matrix with results:

\tiny

\begin{verbatim}
#>           Reference
#> Prediction high low
#>       high   23   0
#>       low     7  26
\end{verbatim}

\normalsize

The following result yields a model with 85.71\% accuracy. The model is
overall quite accurate given the small size of the training set in terms
of total amount of observations. The specificity and sensitivity are
both quite decent as well, hovering \textasciitilde{}85\%.

\newpage

\hypertarget{predictive-power-per-variable}{%
\section{Predictive power per
variable}\label{predictive-power-per-variable}}

We only plot the variables that show most strength when predicting:

\includegraphics{./figures/unnamed-chunk-8-1.pdf}

\hypertarget{variables-that-classify-the-best}{%
\subsection{Variables that classify the
best}\label{variables-that-classify-the-best}}

Now, we can plot each variable kernel density, distinguishing the
categories, to see which variables are the most helpful at predicting
each HDI category.

In other words, we can inspect every plot and see what variables behave
differently for each category and thus, will be more helpful predicting
the category of the observations.

The following are the variables that most help us in categorizing:

\begin{verbatim}
#> [1] "foreign_inv_inflows"   "int_tourism_arrivals"  "int_tourism_receipts" 
#> [4] "perc_internet_users"   "access_to_electricity" "birth_rate"           
#> [7] "mobile_subscriptions"  "infant_mort_rate"      "urban_pop_perc"
\end{verbatim}

For example, we can clearly see that ``access to electricity'' is a
feature that really helps us in distinguishing between the categories.
In general, we could say that countries with a low percentage of the
population with access to electricity are more inclined to be
categorized as a low HDI country, while if an observation (a country)
has a large percentage of ``access to electricity'' it is more inclined
to be categorized as a high HDI country.

On the other hand, the variable ``birth\_rate'' is also important, but
in a different direction: if the country presents a low birth\_rate it
is more inclined to be categorized as high HDI, while a high
``birth\_rate'' presented by a country is a sign of inclination towards
a low HDI country.

\hypertarget{critics}{%
\subsection{Critics}\label{critics}}

As we can see from our description of the model, the two assumptions we
take are obviously not correct in the case of our features.

First, our features are not independent one from the other, meaning that
there is some kind of dependancy between them. It is pretty obvious that
the demographic and economical factors of a country are in some way
dependant one from the other. For example, we can think that a country
with a high infant mortality will probably have a low level of
education, or in a country with a lot of inflation there will not be
very high levels of savings (as the savings would lose value with time
if the inflation persists).

Second, as we just showed with the plots explantions there are some
variables that clearly have a much bigger impact that others in
differentiationg the categories of our response variable. In other
words: we can say that the birth rate of a country contributes more to
predict wether this country is of low or high HDI than, for instance,
the agricultural land.

\newpage

\hypertarget{appendix-code}{%
\section{Appendix: Code}\label{appendix-code}}

\begin{Shaded}
\begin{Highlighting}[]
\CommentTok{# importing libraries}
\KeywordTok{library}\NormalTok{(MASS)}
\KeywordTok{library}\NormalTok{(dplyr)}
\KeywordTok{library}\NormalTok{(e1071)}
\KeywordTok{library}\NormalTok{(mlbench)}
\KeywordTok{library}\NormalTok{(caret)}

\CommentTok{# reading the data into a dataframe}
\NormalTok{data <-}\StringTok{ }\KeywordTok{read.csv}\NormalTok{(}\StringTok{'./data/data.csv'}\NormalTok{)}

\CommentTok{# removing columns that aren't useful}
\NormalTok{data <-}\StringTok{ }\NormalTok{data[}\KeywordTok{names}\NormalTok{(data) }\OperatorTok{!=}\StringTok{ "hdi"} \OperatorTok{&}\StringTok{ }
\StringTok{             }\KeywordTok{names}\NormalTok{(data) }\OperatorTok{!=}\StringTok{ "X"} \OperatorTok{&}\StringTok{ }
\StringTok{             }\KeywordTok{names}\NormalTok{(data) }\OperatorTok{!=}\StringTok{ "country_name"} \OperatorTok{&}\StringTok{ }
\StringTok{             }\KeywordTok{names}\NormalTok{(data) }\OperatorTok{!=}\StringTok{ "country_code"} \OperatorTok{&}\StringTok{ }
\StringTok{             }\KeywordTok{names}\NormalTok{(data) }\OperatorTok{!=}\StringTok{ "year_code"} \OperatorTok{&}\StringTok{ }
\StringTok{             }\KeywordTok{names}\NormalTok{(data) }\OperatorTok{!=}\StringTok{ "year"}\NormalTok{]}

\CommentTok{# changing the very high, medium categories to }
\CommentTok{# high and low respectively}
\NormalTok{cat <-}\StringTok{ }\KeywordTok{c}\NormalTok{()}
\ControlFlowTok{for}\NormalTok{ (i }\ControlFlowTok{in} \DecValTok{1}\OperatorTok{:}\KeywordTok{length}\NormalTok{(data}\OperatorTok{$}\NormalTok{hdi_cat)) \{}
    \ControlFlowTok{if}\NormalTok{ (data}\OperatorTok{$}\NormalTok{hdi_cat[i] }\OperatorTok{==}\StringTok{ "very high"}\NormalTok{) \{}
\NormalTok{        cat <-}\StringTok{ }\KeywordTok{c}\NormalTok{(cat, }\StringTok{"high"}\NormalTok{)}
\NormalTok{    \} }\ControlFlowTok{else} \ControlFlowTok{if}\NormalTok{ (data}\OperatorTok{$}\NormalTok{hdi_cat[i] }\OperatorTok{==}\StringTok{ "medium"}\NormalTok{) \{}
\NormalTok{        cat <-}\StringTok{ }\KeywordTok{c}\NormalTok{(cat, }\StringTok{"low"}\NormalTok{)}
\NormalTok{    \} }\ControlFlowTok{else}\NormalTok{ \{}
\NormalTok{        cat <-}\StringTok{ }\KeywordTok{c}\NormalTok{(cat, data}\OperatorTok{$}\NormalTok{hdi_cat[i])}
\NormalTok{    \}}
\NormalTok{\}}

\CommentTok{# replacing the original hdi_cat col with}
\CommentTok{# the modified version}
\NormalTok{data}\OperatorTok{$}\NormalTok{hdi_cat <-}\StringTok{ }\NormalTok{cat}

\CommentTok{# train-test split}
\NormalTok{n=}\KeywordTok{nrow}\NormalTok{(data)}
\NormalTok{trainset=(}\DecValTok{1}\OperatorTok{:}\NormalTok{n)}\OperatorTok\KeywordTok{sample}\NormalTok{(n,}\KeywordTok{floor}\NormalTok{(n}\OperatorTok{*}\FloatTok{0.7}\NormalTok{)) }
\NormalTok{testset=}\OperatorTok{!}\NormalTok{trainset }

\CommentTok{# naive bayes model}
\NormalTok{model <-}\StringTok{ }\KeywordTok{naiveBayes}\NormalTok{(hdi_cat }\OperatorTok{~}\StringTok{ }\NormalTok{., }\DataTypeTok{data =}\NormalTok{ data,}\DataTypeTok{subset =}\NormalTok{ trainset)}

\CommentTok{# predict and classification of probabilities}
\NormalTok{pred =}\StringTok{ }\KeywordTok{predict}\NormalTok{(model,data[testset,],}\DataTypeTok{type=}\StringTok{"raw"}\NormalTok{)}
\NormalTok{pred_cat <-}\StringTok{ }\KeywordTok{ifelse}\NormalTok{(pred[,}\DecValTok{1}\NormalTok{] }\OperatorTok{>}\StringTok{ }\FloatTok{0.9}\NormalTok{, }\StringTok{"high"}\NormalTok{, }\StringTok{"low"}\NormalTok{)}

\CommentTok{# adding 2 columns to the prediction table}
\CommentTok{# to create a confidence interval for each}
\CommentTok{# prediction}
\NormalTok{num_highs <-}\StringTok{ }\KeywordTok{table}\NormalTok{(pred_cat)[}\DecValTok{1}\NormalTok{]}
\NormalTok{pred <-}\StringTok{ }\KeywordTok{cbind}\NormalTok{(pred, }\KeywordTok{rep}\NormalTok{(}\DecValTok{0}\NormalTok{,}\DecValTok{56}\NormalTok{))}
\NormalTok{pred <-}\StringTok{ }\KeywordTok{cbind}\NormalTok{(pred, }\KeywordTok{rep}\NormalTok{(}\DecValTok{0}\NormalTok{,}\DecValTok{56}\NormalTok{))}
\KeywordTok{colnames}\NormalTok{(pred) <-}\StringTok{ }\KeywordTok{c}\NormalTok{(}\StringTok{"high"}\NormalTok{,}\StringTok{"low"}\NormalTok{,}\StringTok{"low_CI"}\NormalTok{, }\StringTok{"high_CI"}\NormalTok{)}

\CommentTok{# performing the CI calculation}
\ControlFlowTok{for}\NormalTok{ (i }\ControlFlowTok{in} \DecValTok{1}\OperatorTok{:}\KeywordTok{length}\NormalTok{(pred_cat)) \{}
\NormalTok{    p <-}\StringTok{ }\NormalTok{pred[i,}\DecValTok{1}\NormalTok{]}
\NormalTok{    val <-}\StringTok{ }\KeywordTok{qnorm}\NormalTok{(}\FloatTok{0.975}\NormalTok{)}\OperatorTok{*}\KeywordTok{sqrt}\NormalTok{((p}\OperatorTok{*}\NormalTok{(}\DecValTok{1}\OperatorTok{-}\NormalTok{p))}\OperatorTok{/}\NormalTok{num_highs)}
\NormalTok{    pred[i,}\DecValTok{3}\NormalTok{] =}\StringTok{ }\NormalTok{p}\OperatorTok{-}\NormalTok{val}
\NormalTok{    pred[i,}\DecValTok{4}\NormalTok{] =}\StringTok{ }\NormalTok{p}\OperatorTok{+}\NormalTok{val}
\NormalTok{\}}

\CommentTok{# converting the columns to factors}
\NormalTok{pred <-}\StringTok{ }\KeywordTok{as.factor}\NormalTok{(pred_cat)}
\NormalTok{data}\OperatorTok{$}\NormalTok{hdi_cat <-}\StringTok{ }\KeywordTok{as.factor}\NormalTok{(data}\OperatorTok{$}\NormalTok{hdi_cat)}

\CommentTok{# displaying a confusion matrix to assess model quality}
\KeywordTok{confusionMatrix}\NormalTok{(pred, data}\OperatorTok{$}\NormalTok{hdi_cat[testset])}

\CommentTok{# assessing the quality of each variable at predicting}
\NormalTok{mms=}\KeywordTok{apply}\NormalTok{(data[}\DecValTok{1}\OperatorTok{:}\DecValTok{19}\NormalTok{],}\DecValTok{2}\NormalTok{,}\ControlFlowTok{function}\NormalTok{(x) }\KeywordTok{unlist}\NormalTok{(}\KeywordTok{by}\NormalTok{(x,data}\OperatorTok{$}\NormalTok{hdi_cat,mean)))}
\NormalTok{sds=}\KeywordTok{apply}\NormalTok{(data[}\DecValTok{1}\OperatorTok{:}\DecValTok{19}\NormalTok{],}\DecValTok{2}\NormalTok{,}\ControlFlowTok{function}\NormalTok{(x) }\KeywordTok{unlist}\NormalTok{(}\KeywordTok{by}\NormalTok{(x,data}\OperatorTok{$}\NormalTok{hdi_cat,sd)))}

\CommentTok{# plotting the kernel densities}
\KeywordTok{par}\NormalTok{(}\DataTypeTok{mfrow=}\KeywordTok{c}\NormalTok{(}\DecValTok{3}\NormalTok{,}\DecValTok{2}\NormalTok{))}
\ControlFlowTok{for}\NormalTok{(i }\ControlFlowTok{in} \DecValTok{1}\OperatorTok{:}\DecValTok{19}\NormalTok{)\{}
\NormalTok{  rrx=}\KeywordTok{range}\NormalTok{(}\KeywordTok{c}\NormalTok{(mms[,i]}\OperatorTok{+}\DecValTok{3}\OperatorTok{*}\NormalTok{sds[,i],mms[,i]}\OperatorTok{-}\DecValTok{3}\OperatorTok{*}\NormalTok{sds[,i]))}
\NormalTok{  rry=}\KeywordTok{c}\NormalTok{(}\DecValTok{0}\NormalTok{,}\KeywordTok{max}\NormalTok{(}\KeywordTok{dnorm}\NormalTok{(mms[,i],mms[,i],sds[,i]}\OperatorTok{+}\FloatTok{0.0001}\NormalTok{)))}
  \KeywordTok{plot}\NormalTok{(}\DecValTok{0}\NormalTok{,}\DecValTok{1}\NormalTok{,}\DataTypeTok{type=}\StringTok{"n"}\NormalTok{,}\DataTypeTok{xlab=}\KeywordTok{colnames}\NormalTok{(mms)[i],}\DataTypeTok{ylab=}\StringTok{"Normal Density"}\NormalTok{,}\DataTypeTok{xlim=}\NormalTok{rrx,}\DataTypeTok{ylim=}\NormalTok{rry)}
\NormalTok{  ss=}\KeywordTok{seq}\NormalTok{(rrx[}\DecValTok{1}\NormalTok{],rrx[}\DecValTok{2}\NormalTok{],}\DataTypeTok{length.out =} \DecValTok{100}\NormalTok{)}
  \ControlFlowTok{for}\NormalTok{(j }\ControlFlowTok{in} \DecValTok{1}\OperatorTok{:}\DecValTok{2}\NormalTok{) }\KeywordTok{points}\NormalTok{(ss,}\KeywordTok{dnorm}\NormalTok{(ss,mms[j,i],sds[j,i]),}\DataTypeTok{type=}\StringTok{"l"}\NormalTok{,}\DataTypeTok{col=}\NormalTok{j)}
\NormalTok{\}}

\CommentTok{# good columns, variables that are solid at predicting }
\CommentTok{# HDI category}
\NormalTok{good_cols <-}\StringTok{ }\KeywordTok{c}\NormalTok{(}\StringTok{"foreign_inv_inflows"}\NormalTok{, }\StringTok{"int_tourism_arrivals"}\NormalTok{, }\StringTok{"int_tourism_receipts"}\NormalTok{, }\StringTok{"perc_internet_users"}\NormalTok{, }\StringTok{"access_to_electricity"}\NormalTok{, }\StringTok{"birth_rate"}\NormalTok{, }\StringTok{"mobile_subscriptions"}\NormalTok{, }\StringTok{"infant_mort_rate"}\NormalTok{, }\StringTok{"urban_pop_perc"}\NormalTok{)}
\NormalTok{good_cols}
\end{Highlighting}
\end{Shaded}

\end{document}
